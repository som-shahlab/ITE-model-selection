\section{AUC-style selection methods}

\label{other}

There are a few treatment effect model selection approaches that we have not examined: the direct marketing literature documents the use of various functions of the ``uplift curve'' \cite{Gutierrez:2016tq}, while \citet{vanKlaveren:2018gg} propose what they call the concordance-for-benefit statistic. 

Recall that the decision policy can be parametrized with a cutoff $k$: $\hat d(x,k) = I(\hat\tau>k)$. We can therefore also calculate the gain at $k$, $\hat g(k)$, which is the gain obtained when individuals with an estimated treatment effect greater than $k$ are treated. Researchers in direct marketing often rely on \emph{uplift curves} (also called gain curves or cumulative gain charts) \ to aid in model selection. The uplift curve plots gain at $k$ against the percentage of individuals in the validation set recommended for treatment at $k$. These curves are evaluated either heuristically, using the maximum value over $k$, or, most commonly, using the area under the curve \cite{Gutierrez:2016tq}. The area under the gain curve is called the \emph{Qini coefficient}.

\citet{vanKlaveren:2018gg} propose the concordance-for-benefit statistic to select among treatment effect models. Briefly, each individual $i$ in the validation set is matched without replacement according to estimated treatment effect with another individual in the validation set $\bar{i}$ ($\hat\tau(x_i) \approx \hat\tau(x_{\bar i})$) with the opposite treatment $w_i \ne w_{\bar i}$. The total number of pairs is $J$. For each pair of matched patients $j$, the difference in outcomes between the treated and untreated individuals is recorded: $\delta_j = (2w_i - 1)(y_i -y_{\bar i})$, along with the predicted treatment effect for both patients $t_j = \hat\tau(x_i)$. The concordance-for-benefit (or c-for-benefit) statistic is

\[
\hat c = \frac{\sum_j^J \sum_{j' > j}^J (1 - \text{sign}((\delta_{j'} - \delta_j)(t_{j'} - t_j))) }{(J^2-J)}
\]

This is the proportion of pairs-of-pairs in which the difference in predicted treatment effects is discordant with the difference in ``observed'' treatment effects in the validation set. This assesses the overall ranking of individuals with respect to their estimated treatment effects.

Both of these approaches are inspired by different interpretations of the area under the receiver operating characteristic curve statistic (AUC), which is both the area under a parametric curve indexed by a classification cutoff (i.e. $k$) and an estimate of the proportion of pairs expected to have discordant classifications \cite{Hanley:1982cz}. We conjecture that there is a theoretical relationship between these approaches and that they are related to decision value. We test both of these methods in simulation, along with value-at-k AUC, which is the same as the Qini coefficient, but using $\hat{v(k)}$ instead of $\hat{g(k)}$.

%%%%%%%%%%%%%%%%%%%%%%%%%%%%%%%%%%%%%%%%%%%%%%%%%%%%%%%%%%%%%%%%%%%%%%%%%%%%%%%%%

\begin{comment}

Recall that the decision policy can be parametrized with a cutoff $k$: $\hat d(x,k) = I(\hat\tau>k)$. We can therefore also calculate the gain at $k$, $\hat g(k)$, which is the gain obtained when individuals with an estimated treatment effect greater than $k$ are treated. Researchers in direct marketing often rely on \emph{uplift curves} (also called gain curves or cumulative gain charts) \ to aid in model selection. The uplift curve plots gain at $k$:

\[
	\hat g(k)  =  \sum_{\mathcal{V}} \dfrac{y_i  \hat d(x_i,k) (2w_i-1)}{p_{w_i}(x_i)} 
\]




Researchers either evaluate these curves heuristically, take the maximum value over $k$, or calculate the \emph{Qini coefficient}, which is the area under the uplift curve \cite{Gutierrez:2016tq}. Given the relationship between gain and value, maximizing these metrics is equivalent to maximizing the equivalent value-based metric. 

\subsubsection{The concordance-for-benefit statistic}

\citet{vanKlaveren:2018gg} propose the concordance-for-benefit statistic to select among treatment effect models. Briefly, each individual $i$ in the test set is matched without replacement according to estimated treatment effect with another individual in the test set $\bar{i}$ ($\hat\tau(x_i) \approx \hat\tau(x_{\bar i})$) with the opposite treatment $w_i \ne w_{\bar i}$. The total number of pairs is $J$. For each pair of matched patients $j$, the difference in outcomes between the treated and untreated individuals is recorded: $\delta_j = (2w_i - 1)(y_i -y_{\bar i})$, along with the predicted treatment effect for both patients $t_j = \hat\tau(x_i)$. The concordance-for-benefit (or c-for-benefit) statistic is

\[
\hat c = \frac{\sum_j^J \sum_{j' > j}^J (1 - \text{sign}((\delta_{j'} - \delta_j)(t_{j'} - t_j))) }{(J^2-J)}
\]

This is the proportion of pairs-of-pairs in which the difference in predicted treatment effects is discordant with the difference in ``observed'' treatment effects in the test set. 

At first glance, this resembles the ineffective treatment effect matching approach we describe in section \ref{te-match}. The difference is that the c-for-benefit does not assess the disparity between ``observed'' and estimated treatment effects- it assesses the quality of the ranking according to the estimated treatment effect. The c-for-benefit is an estimate of:

\[
P(Y_i - Y_{\bar i} > Y_{i'} - Y_{\bar i'} | W_i = W_{i'} = 1, W_{\bar i} = W_{\bar i'} = 1, \hat\tau(X_i) = \hat\tau(X_{\bar i}), \hat\tau(X_{i'}) = \hat\tau(X_{\bar i'}), \hat\tau(X_i) > \hat\tau(X_{i'}))
\]

The c-for-benefit may be related to value and gain. To see this, we simplify the procedure: instead of matching pairs, we bin individuals into $J$ bins, each with approximately equal treatment effect: $S_j = \{i | l_j \le \hat\tau(x_i) < u_j\}$ where $u_j = l_{j+1}$, $l_1 = \underset{i \in \mathcal{V}}{\text{min}} \ \hat\tau(x_i)$, and $u_J = \underset{i \in \mathcal{V}}{\text{max}} \ \hat\tau(x_i)$. To capture the spirit of the original definition, we let $t_j$ be the average estimated treatment effect in each bin and we calculate 

\[
\delta_j = \sum_{i \in S_j} \frac{y_i w_i}{p_i} - \sum_{i \in S_j} \frac{y_i (1-w_i)}{(1-p_i)}
\]

so that $\delta_j$ is an unbiased estimate of the treatment effect among individuals in bin $S_j$. Note that this definition of $\delta_j$ is identical to our definition of the generalized gain $\hat g$ if the policy is $\hat d(x_i) = I(i \in S_j)$. In the case where there are only two bins $S_1 = \{i | \hat\tau(x_i) < 0\}$ and $S_2 = \{i | 0 \le \hat\tau(x_i)\}$, the binned c-for-benefit statistic is

\[
\hat c^{^{split}} = \frac{1}{2} \left( 1 - \text{sign}((\delta_2 - \delta_1)(t_2 - t_1)) \right)
\]

By construction, $t_2 - t_1 > 0$. We also have that $\delta_1 = \hat g(1-\hat d(x))$ and $\delta_2 = \hat g(\hat d(x))$, where $\hat d(x) = I(\hat\tau(x) > 0)$. Therefore

\[
\hat c^{split} = \frac{1}{2} (1 - \text{sign}(\hat g(\hat d) - \hat g(1- \hat d)))
\]

\begin{theorem} 
As the size of the test set goes to infinity, the model that maximizes gain (and thus value) is one of the models that minimizes $\hat c^{split}$.
\end{theorem}

\begin{proof}
Recall that $E[\hat g(\hat d)] = E[\tau(X) \hat d(X)]$. Now note that
\[
\begin{array}{rcl}
	E[\hat g(1-\hat d)] &=& E[E[\hat g(1-\hat d)|X]] \\
	&=&  E[\tau(X)(1-\hat d(X))] \\
	&=&  E[\tau(X) - \tau(X) \hat d(X)] \\
	&=&  \bar \tau - E[\tau(X) \hat d(X)] \\
	&=&  \bar \tau - E[\hat g(\hat d)]
\end{array}
\]	
	
Take two decision rules $\hat d_a$ and $\hat d_b$. 
\[
\begin{array}{rcl}
E[\hat c^{split}_a - \hat c^{split}_b] & = & - E[\text{sign}(\hat g_a - \hat g_{1-a})] + E[\text{sign}(\hat g_b - \hat g_{1-b})] \\
\end{array}
\] 

\emph{*waves hands*} In the limit, $\hat g_m - \hat g_{1-m}$ should converge to $2E[\hat g_m] - \bar \tau$, so we have the following cases:

\[
E[\hat c^{split}_a - \hat c^{split}_b] \rightarrow \left\{
\begin{array}{rcl}
2 & \text{if} & E[\hat g_b] > \bar \tau \text{ and } E[\hat g_a] < \bar \tau \\
0 & \text{if} & E[\hat g_b] > \bar \tau \text{ and } E[\hat g_a] > \bar \tau \\
0 & \text{if} & E[\hat g_b] < \bar \tau \text{ and } E[\hat g_a] < \bar \tau \\
-2 & \text{if} & E[\hat g_b] < \bar \tau \text{ and } E[\hat g_a] > \bar \tau \\
\end{array} \right.
\] 

Consequently, $E[\hat c^{split}_a] > E[\hat c^{split}_b]$ only if $E[\hat g_a] < E[\hat g_b]$ and $E[\hat c^{split}_a] < E[\hat c^{split}_b]$ only if $E[\hat g_a] > E[\hat g_b]$.

\end{proof}

To illustrate this relationship we have had to simplify the c-for-benefit, which may fundamentally change its nature. The original pair-matching procedure is adaptive, whereas our binned procedure uses an a-priori split.

\end{comment}