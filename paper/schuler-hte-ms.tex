\documentclass{article}

% FOR ARXIV:
\pdfoutput=1

% if you need to pass options to natbib, use, e.g.:
% \PassOptionsToPackage{numbers, compress}{natbib}
% before loading nips_2017
%
% to avoid loading the natbib package, add option nonatbib:
%\usepackage[nonatbib]{nips_2017}
% to compile a camera-ready version, add the [final] option, e.g.:
%\usepackage[final,nonatbib]{nips_2017}

% \usepackage{nips_2017}
\usepackage[numbers]{natbib}


\usepackage[utf8]{inputenc} % allow utf-8 input
\usepackage[T1]{fontenc}    % use 8-bit T1 fonts
\usepackage{hyperref}       % hyperlinks
\usepackage{url}            % simple URL typesetting
\usepackage{booktabs}       % professional-quality-adjusted-life-yearslity tables
\usepackage{amsfonts}       % blackboard math symbols
\usepackage{nicefrac}       % compact symbols for 1/2, etc.
\usepackage{microtype}      % microtypography
\usepackage{hyperref}
\hypersetup{
    colorlinks=true,
    linkcolor=blue,
    filecolor=magenta,      
    urlcolor=cyan,
    citecolor=red
}
\usepackage{bm}
\usepackage{tabu}
\usepackage{comment}
\usepackage{amsmath} 
\usepackage{subcaption}
\usepackage{listings} 
\usepackage{graphicx}
\usepackage{caption}
\usepackage{rotfloat}
\usepackage{float}
\usepackage{amsthm}
\usepackage{array}
\usepackage[linesnumbered, lined, ruled, boxed, commentsnumbered]{algorithm2e}
\graphicspath{{figures/}}
\newtheorem{theorem}{Theorem}
\newtheorem{lemma}{Lemma}
\newcommand\independent{\protect\mathpalette{\protect\independenT}{\perp}}
\def\independenT#1#2{\mathrel{\rlap{$#1#2$}\mkern2mu{#1#2}}}

\captionsetup[table]{skip=10pt}

\begin{document}

\title{General-purpose model selection when estimating individual treatment effects}

\author{Alejandro Schuler \\
	aschuler@stanford.edu \\
       Biomedical Informatics Research\\
       Stanford University\\
       Palo Alto, CA, USA 
          \and
     Nigam Shah \\
     nigam@stanford.edu \\
       Biomedical Informatics Research\\
       Stanford University\\
       Palo Alto, CA, USA 
                 \and
     Michael Baiocchi \\
     baiocchi@stanford.edu \\
       Stanford University\\
       Palo Alto, CA, USA}

\maketitle

\begin{abstract}
{Practitioners in medicine, business, political science, and other fields are increasingly aware that decisions should be personalized to each patient, customer, or voter. A given treatment (e.g. a drug or advertisement) should be administered only to those who will respond most positively, and certainly not to those who will be harmed by it. Individual-level treatment effects can be estimated with tools adapted from machine learning, but different models can yield contradictory estimates. Unlike risk prediction models, however, treatment effect models cannot be easily evaluated against each other using a held-out test set because the true treatment effect itself is never directly observed. Besides outcome prediction accuracy, several metrics that can leverage held-out data to evaluate treatment effects models have been proposed, but they are not widely used. We provide a didactic framework that elucidates the relationships between the different approaches and compare them all using a variety of simulations of both randomized and observational data.  We find that the R-objective is the best performing in a variety of settings.}%{Individual treatment effects; heterogenous treatment effects; model selection; model validation; causal inference.}
\end{abstract} 

\section{Introduction}

The general decision problem we address is as follows: for a particular individual, a decision maker must choose between prescribing an intervention or no intervention. The intervention (treatment) may be a drug, an advertisement, a campaign email etc. The decision-maker's goal is to maximize some outcome for that patient or customer, which may be their lifespan, their net purchases, their political engagement etc. This is a causal inference problem because we seek to discover a causal relationship between the intervention and outcome. The causal \emph{treatment effect} is the difference between what would have happened had the individual been given the intervention and what would of happened had they not been. The outcomes under these different scenarios are referred to as \emph{potential outcomes} (\citealp{Rubin2005}).

Prior to the development of modern statistical methods, treatment policies were generally one-size-fits-all prescriptions based on estimates of the average treatment effect (\citealp{Segal:ub}). Experiments for inferring average effects limit individual heterogeneity by imposing strict criteria on the population under study (\citealp{Stuart:2014id}). Recently, however, researchers in multiple domains have attempted to leverage modern statistical technology and real-world data to tailor decisions to individuals; this phenomena is exemplified by the rise of personalized medicine (\citealp{Ferreira:2017fv}) and targeted advertisement (\citealp{Ascarza:2018ie, Matz:2017ix}). Decision-makers recognize that treating to the average, while expedient, does not result in the best outcome for all individuals (\citealp{Kravitz:2004fa,Segal:ub}). Attention is increasingly focused on the estimation of individual treatment effects.
\footnote{Note that estimating individual treatment effects is not the same as estimating personalized risks or prognoses with prediction models (e.g. for a heart attack, customer churn, or non-voting). Prediction models only predict what would happen to the individual given standard practice, not the difference of what would happen if a treatment were or were not given. As such, prediction models by themselves are often of little practical utility unless the effects of available treatments are known and relatively constant. If that is not the case, targeting treatment to individuals at high-risk for the outcome is not an optimal strategy: there may be high-risk individuals who do not respond or respond negatively to the treatment, and low-risk individuals who would respond very positively (\citealp{Ascarza:2018ie}).}

To clarify further discussion, we characterize each individual $i$ by a vector of pre-treatment features or covariates $x_i$, their treatment status (intervention $w_i=1$ or no intervention $w_i=0$) and their outcome $y_i$. Using the potential outcomes framework (\citealp{Rubin2005}), we write the potential outcomes under treatment and control as $Y(1)$ and $Y(0)$ and their conditional means as $\mu_0(x) = E[Y(0)|X=x]$ and $\mu_1(x) = E[Y(1)|X=x]$. The estimand in question is the conditional average treatment effect $\tau(x) = \mu_1(x) - \mu_0(x)$, which is the expected difference in potential outcomes under the alternative interventions for the individual in question. In different fields this quantity is alternatively called the individual treatment effect, individual causal effect, individual benefit, or individual lift. If the true conditional mean functions $\mu_w(x)$ are known, the rule (policy) that assigns each individual $x_i$ their optimal treatment is $d(x) = \underset{w \in \{0,1\}}{\text{argmax}} \ \ \mu_w(x)$, or, alternatively, $d(x) = I(\tau(x) > 0)$ where $I$ is the indicator function. Generally, the conditional mean functions $\mu_w$ are unknown, meaning that there is uncertainty about the individual treatment effect and optimal treatment policy.

There currently exist a number of methods to estimate individual treatment effects from randomized data. The process of estimating these effects is alternatively referred to as heterogenous effect modeling, uplift modeling, or individual treatment effect modeling. These approaches can also be used for observational data if certain assumptions are met or if combined with propensity score or matching techniques.

A manual subgroup analysis is the traditional approach to heterogenous treatment effect estimation. A subgroup analysis partitions the population of individuals into manually-specified subgroups and typically estimates an average treatment effect in each subgroup using traditional methods (i.e. linear or logistic regression) (\citealp{Gail:1985ft}). This approach requires a high degree of domain knowledge is prone to multiple-hypothesis testing problems if subgroups are not pre-specified. 

An alternative is to use any supervised learning method (e.g. LASSO, random forest, neural network) to fit functions $\hat\mu_0$ and $\hat\mu_1$ that estimate the conditional means $\mu_0$ and $\mu_1$ of the potential outcomes. These estimates are then used to estimate the treatment effect $\hat\tau(x) = \hat\mu_1(x) - \hat\mu_0(x)$ (\citealp{Gutierrez:2016tq, Austin:2012cy, Snowdn:2011ef}). This can be done by regressing the observed outcomes on the covariates in the untreated group to get $\hat\mu_0$ and regressing the observed outcomes on the covariates in the treated group to get $\hat\mu_1$. \citet{Kunzel:2017vg} call this approach T-learning (T for ``two models''). Similarly, it is possible to fit a single model $\hat\mu(x,w)$ and estimate the treatment effect in the same way as above by letting $\hat\mu_w(x) = \hat\mu(x,w)$ (S-learning, for single-model) \citealp{Kunzel:2017vg}. T-learning and S-learning together have been referred to as simulated twins, g-computation, counterfactual regression, or conditional mean regression. 

Modeling the conditional means is a valid approach, but many have noted that since the object of interest is the treatment effect we may be better off modeling it directly without appeal to the correctness of $\hat\mu_0$ and $\hat\mu_1$. Approaches in this vein include \citet{Zhao:2017vi}, \citet{Athey:2016wm}, \citet{Powers:2017wd}, and \citet{Nie:2017vi}. 

Among the variety of approaches and the number of hyper-parameter settings within each approach, which is best? As is the case with all statistical learning problems, there is no context-free way of knowing (\citealp{Wolpert:1996fp}): different methods will give better or worse estimates depending on the application. Indeed, using a large set of diverse simulations, \citet{Dorie:2017uo} find that the only somewhat consistent predictor of the success of a causal inference method is its ability to ``flexibly'' model the conditional means or treatment effect. Although that result surely depends upon the particulars of their simulations, it parallels the common knowledge in the machine learning literature that deep nets and additive regression trees often outperform linear models for real-world applications. However, even limiting ourselves to flexible treatment effect modeling methods, we are left with a panoply of approaches and hyper-parameter settings to chose from. 

We digress briefly to discuss the standard supervised learning setting where the task is to estimate $y$ given $x$ by building an estimator $\hat{\mu}(x)$. In this setting we can use the diversity of machine learning approaches to our advantage by performing model selection. Given $M$ modeling approaches and/or hyper-parameter settings, we build $M$ estimators $\{\hat \mu_1, \hat \mu_2 \dots \hat \mu_m \dots \hat \mu_M\}$. The quantity of interest in this case is the expected prediction risk of the model when it is applied to new data, according to some loss $l$. We express that as $E[l(\hat \mu_m(X), Y)]$. The idea of model selection is to estimate this expected risk for each of our models and find the model that minimizes it. There are several ways to estimate this risk, including information criteria, but data-splitting is the easiest and most widely applicable (\citealp{esl:2009wc, Arlot:2010fl, Dudoit:2005jw}). Before fitting the models, the observations are randomly split into training and validation samples. The models are fit on the training sample $\mathcal{T}$ and evaluated on the validation sample $\mathcal{V}$. The risk of each model is estimated with $\frac{1}{|\mathcal{V}|}\sum_{i \in \mathcal{V}}^{|\mathcal{V}|} l(\hat \mu_m (x_i), y_i)$. In cross-validation, this process is repeated round-robin across different random splits of the data and the estimated risks are averaged per model.

This approach breaks down for treatment effect estimation because the true treatment effect is never observed in any sample. In this case, the quantity of interest is the $\tau$-risk: $E[l(\hat\tau, \tau)]$. We would like to evaluate models $\{\hat\tau_1, \hat\tau_2 \dots \hat \tau_m \dots \hat \tau_M\}$ by estimating their $\tau$-risk on a validation set via $\frac{1}{|\mathcal{V}|}\sum_{i \in \mathcal{V}}^{|\mathcal{V}|}  l(\hat \tau_m (x_i), \tau(x_i))$ (this quantity has been called the precision in estimating heterogenous effects or PEHE by \citet{Hill2011}). The problem is that we never observe $\tau(x_i)$ directly (we only see one of the two potential outcomes) and thus have nothing to compare $\hat\tau(x_i)$ to. This estimator of $\tau$-risk is thus infeasible.

Several treatment effect model selection approaches have been suggested in the literature, but none of them enjoy the wide use and dominance that prediction error cross-validation has in the supervised learning setting. Most approaches are not general-purpose in that they require that the set of estimators $\{\hat\tau_1, \hat\tau_2 \dots \hat \tau_m \dots \hat \tau_M\}$ come from a specific class of models. For example, \citet{Powers:2017wd} and \citet{Athey2015} both use selection methods that are specific to the models they propose. The Focused Information Criterion (FIC) (\citealp{Claeskens:2003ck}) is a promising approach, but as of yet cannot be used to select between most machine learning estimators (\citealp{Jullum:2012uo}). \citet{Alaa:tj} propose an empirical Bayes approach for optimal prior selection.

It is clear that we lack a go-to general-purpose approach for applied researchers to select among treatment effect models, leaving the door open to poor practice. Using valid model selection will ensure that better models result from primary research, making it more certain, for instance, that a patient will benefit from their treatment or that an advertisement will reach interested parties. 

In this work, we summarize several model selection approaches that use an independent validation set to judge the quality of individual treatment effects estimated from a training set. Some of these approaches have been explicitly proposed for model selections- the rest are adapted from model fitting procedures proposed for individual treatment effect estimation or policy learning. We implement each of these approaches and test them against each other in a variety of experimental simulations where the usual assumptions of positivity, SUTVA, and ignorability hold. An outline of the paper is as follows. In section \ref{approaches}, we describe approaches for treatment effect model selection and introduce a didactic framework to relate them to each other. Section \ref{simulations} describes our experiments and results. We conclude in section \ref{conclusion} with a summary of our contributions and recommendations for researchers interested in estimating individual treatment effects.
 
\section{Existing Approaches}
\label{approaches}

\subsection{Outcome MSE}

There are many simple examples where minimizing the mean-squared error of predicted outcomes badly fails to select the model with the most accurate treatment effect \cite{Rolling:2013kz}. Despite this, in the absence of anything better, prediction error has been used to select among treatment effect models. Assuming the treatment effect model is built by regressing the outcomes onto the covariates and treatment to obtain $\hat\mu_0$ and $\hat\mu_1$, we can estimate prediction error with

\begin{equation}
	\check e = \frac{1}{|\mathcal{V}|} \sum_{i \in \mathcal{V}}^{|\mathcal{V}|}  
	l(\hat \mu_{w_i} (x_i), y_i) 
\label{pred-error}
\end{equation}
 
For individuals in the test set who were treated ($w=1$), we estimate their outcome using the treated model and assess error, and vice-versa for the untreated. At first glance, this does not appear to relate to the quantity in equation \ref{te-error}. However, if the loss is of the form $l(f,y) = g(|f-y|)$ (e.g. quadratic loss), we can expand this expression by subtracting the predicted counterfactual mean outcome $\hat\mu_{1-w_i}(x_i)$ from both arguments of the loss function and multiplying both arguments by $(2w_i -1)$, which is always $1$ or $-1$.

\begin{equation}
	\check e = \frac{1}{|\mathcal{V}|} \sum_{i \in \mathcal{V}}^{|\mathcal{V}|}  
	l( \ 
	\underbrace{(2w_i -1) (\hat\mu_{w_i} (x_i) - \hat\mu_{1-w_i}(x_i))}_{\hat\tau(x_i)}, 
	\underbrace{(2w_i -1) (y_i - \hat\mu_{1-w_i}(x_i))}_{\check\tau_i}
	) 
\label{pred-error-expansion}
\end{equation}

Now we recognize that the first argument $(2w_i -1) (\hat\mu_{w_i} (x_i) - \hat\mu_{1-w_i}(x_i))$ reduces to $\hat\mu_1 (x_i) - \hat\mu_0(x_i) = \hat\tau(x_i)$, which is the treatment effect estimate from our model. The second argument is the difference between the observed outcome $y_i$ and the counterfactual prediction from our model $\hat\mu_{1-w_i}(x_i)$. We can interpret this difference as an estimate of $\tau(x_i)$. 

Comparing equation \ref{pred-error-expansion} to equation \ref{te-error} shows that if we use outcome prediction error to select among treatment effect models, we are ignoring any error in the prediction of the counterfactual outcomes in the test set. 

Another disadvantage of outcome MSE is that it is not actually general-purpose. It can only be applied to modeling algorithms that give us access to estimated potential outcomes $\mu_w(x_i)$. Methods that target the treatment effect directly sometimes do not provide estimates of the potential outcomes.

\subsection{Matched MSE}

\citet{Rolling:2013kz} propose an estimator $\check \tau_i$ based on matched treated and control individuals in the test set. Briefly, for each individual $i$ in the test set they use Mahalanobis distance matching to identify the most similar individual $\bar{i}$ in the test set with the opposite treatment ($w_i \ne w_{\bar i}$) and compute $\check \tau_i = (2w_i -1)(y_i - y_{\bar i})$ as the plug-in estimate of $\tau(x_i)$. 

They prove under general assumptions and a squared-error loss that a more mathematically tractable version of their algorithm has selection consistency, meaning that it correctly selects the best model as the number of individuals goes to infinity. They conjecture that the practical version of the algorithm retains this property.

The downsides of this approach are that Mahalanobis matching scales relatively poorly and matches become difficult to find in high-dimensional covariate spaces.

\subsection{Transformed outcome MSE}

Here we use the notation $Y^{(w)} = Y|(W=w)$ to denote the potential outcomes and $p(X) = E[W|X=x]$ to denote the propensity score. It has long been known that the transformed outcome 

\begin{equation}
	\begin{array}{rcl}
	Y^{\dagger}  & = & Y^{(0)} \frac{W}{p(X)} - Y^{(1)} \frac{1-W}{1-p(X)} \\
	& = &
	\begin{cases}
		-\frac{Y}{1-p(X)} & \text{if} \ W=0 \\
		\frac{Y}{p(X)} & \text{if} \ W=1\\
	\end{cases} \\
	\end{array}
\end{equation}

is, on expectation under standard assumptions, the treatment effect: $E[Y^{\dagger}|X=x] = \tau(x)$. \citet{Gutierrez:2016tq} leverage this fact and propose an estimator $\check \tau_i = Y^{\dagger}_i$ that they show leads to consistent estimation of the generalization error. In randomized and/or controlled experiments, the propensity score is known. In observational settings, a plug-in estimate of the propensity score may be used.

%\begin{proof}
%\end{proof}

\subsection{Policy value}

\citet{Kapelner:3baXYEjR} and \cite{Zhao:2017wa} select among treatment effect models by comparing their estimated decision-theoretic \emph{values}. Each model $\hat\tau_m(x)$ has an associated set of decision rules $\hat d_{m}(x,k) = I(\hat\tau_m(x) > k)$ which indicate which individuals should be treated if we wish to treat all individuals with expected benefit greater than $k$. Generally, we let $k=0$ so that all individuals who stand to benefit are treated. For notational convenience we write $\hat d_m(x) = \hat d_{m}(x,0)$. The \emph{value} of a decision rule $\hat d$ is $v = E[Y|W=\hat d(X)]$. In other words, the value is the expected outcome of an individual when all individuals are treated according to the policy $\hat d$. If larger values of the outcome are more desirable (e.g. lifespan, click-through rate, approval ratings), then the policy that maximizes the value is optimal, and vice-versa. Without loss of generality, we will assume that we are interested in maximizing value. The best possible policy, $d(x) = I(\tau(x) > 0)$, is generally unknown because we do not know the true treatment effect $\tau(x)$. 

As before, we assume that estimators $\hat\tau_m$ have been previously estimated on an independent training set and we now dedicate our attention to data in the test set $\mathcal{V}$. Somewhat remarkably, we will see that the value of a treatment effect model can be estimated without separately estimating the conditional treatment effect in the test set.

\citet{Kapelner:3baXYEjR} and \cite{Zhao:2017wa} propose the same test set estimator for the value of a treatment effect model:

\begin{equation}
\label{value}
\hat v = \frac{1}{|\mathcal{V}|}\sum_{\mathcal{V}} \frac{y_i I(w_i=\hat d(x_i))}{p_{w_i}(x_i)}
\end{equation}

where $p_{w_i}(x_i) = P(W=w_i | X=x_i)$. This is the propensity score if $w_i = 1$ and one minus the propensity score if $w_i = 0$.

In the randomized setting where $p_{w_i}(x_i) = 0.5$, we can imagine that two side-by-side experiments were run, one in which treatments were assigned according to the model ($W = \hat d(X)$) and one in which they were assigned according to the opposite recommendation ($W = 1 - \hat d(X)$). The data in the test set are a concatenation of the data from these two experiments. To estimate the value of our model, we average the outcomes of individuals in the first experiment and ignore the data from the second experiment. This is essentially what the estimator in equation \ref{value} is doing. When $p_{w_i}(x_i) \ne 0.5$, we must appropriately weigh the outcomes according to the probability of treatment to accomplish the same goal. \citet{Kapelner:3baXYEjR} give a similar explanation, but omit the role of the propensity score. \citet{Zhao:2017wa} provide a short proof that $\hat v$ is unbiased for the true value $v = E[Y|W = \hat d(X)]$. 


\subsection{Gain}

The direct marketing literature has in recent years relied on the concept of \emph{gain} to select treatment effect models (which are usually referred to as ``uplift models''). Gain (sometimes also called \emph{lift}) is defined as follows:

\begin{equation}
\label{gain-basic}
	\hat \gamma = \frac{1}{|\mathcal V |} \left(
		  \frac{\sum_{\mathcal{V}} y_i  \hat d(x_i) w_i}{\sum_{\mathcal{V}}  \hat d(x_i) w_i} - 
		  \frac{\sum_{\mathcal{V}} y_i  \hat d(x_i) (1-w_i)}{\sum_{\mathcal{V}}  \hat d(x_i)  (1-w_i)} 
		  \right)
		  \sum_{\mathcal{V}} \hat d(x_i) 
\end{equation}

The first term inside the parentheses is the average outcome among \emph{treated} individuals in the test set who were also recommended for treatment by the model. The second term is the average outcome among \emph{untreated} individuals in the test set who were recommended for treatment by the model. The resulting difference is an estimate of the average treatment effect among individuals recommended for treatment by the model. That number is multiplied by the total number of individuals in the test set recommended for treatment by the model. 

This estimator is actually a special case of

\begin{equation}
\label{gain}
\begin{array}{rcl}
	\hat g & =& \dfrac{1}{|\mathcal V |} \sum_{\mathcal{V}} \dfrac{y_i  \hat d(x_i) w_i}{p(x_i)} - \sum_{\mathcal{V}} \dfrac{y_i  \hat d(x_i) (1-w_i)}{1-p(x_i)} \\
	&=& \dfrac{1}{|\mathcal V |} \sum_{\mathcal{V}} y^{\dagger}_i \hat d(x_i)
\end{array}
\end{equation}

Equation \ref{gain-basic} can be rewritten as:
\[
	 \dfrac{1}{|\mathcal V |} \underbrace{ \frac{ \sum_{\mathcal{V}} \hat d(x_i)}{\sum_{\mathcal{V}}  \hat d(x_i) w_i} }_{1/\hat p}
		  	\sum_{\mathcal{V}} y_i  \hat d(x_i) w_i - 
		\dfrac{1}{|\mathcal V |}  \underbrace{ \frac{ \sum_{\mathcal{V}} \hat d(x_i)}{\sum_{\mathcal{V}}  \hat d(x_i)  (1-w_i)} }_{1/ (1-\hat p)}
		  	\sum_{\mathcal{V}} y_i  \hat d(x_i) (1-w_i) 
\]

The multipliers underbraced above are unbiased estimates of $1/\hat p_{w_i}$ because of the conditional independence of $\hat d(X)$ and $W$. 

Thus the traditional formula for gain is a version of our formula that is suitable for use when $p(x) = p$ is a constant, as is the case in randomized experiments. By and large, the direct marketing literature has focused on randomized data - to our knowledge this is the first time a gain estimator has been constructed for observational data. 

% \section{Theory}
\label{theory}

As a reminder, we are examining estimators of the form

\begin{equation}
\tag{\ref{te-error}}
\check e = \frac{1}{|\mathcal{V}|}\sum_{i \in \mathcal{V}} l(\hat \tau (x_i), \check \tau_i)
\end{equation}

where $\mathcal V$ is a validation set and $\hat\tau$ is a function estimated from a training set. We are left to choose the estimator $\check\tau_i$ and a loss $l$.

\subsection{Connection between gain, value, and estimated treatment effect estimation error}
\label{sec:gain-value}

It is straightforward to see how the, matching MSE and transformed outcome MSE are special cases of equation \ref{te-error} using squared-error loss and two different estimators for $\check\tau$. Outcome MSE can be framed as an ``incorrect'' case where $\check\tau$ relies on $\hat\tau$ and thus incorporates information from the training set. Here we show that gain is also a special case of equation \ref{te-error} and that maximizing gain is equivalent to maximizing value in expectation.

\begin{theorem}
\label{gain-value}
In expectation, a model that maximizes estimated gain also maximizes estimated value.
\end{theorem}

\begin{proof}
Note that
\[
\begin{array}{rcl}	
	E[\hat g | X=x] & = & E[Y^{\dagger} \hat d(x) | X=x]  \\
	& =  & (\mu_1(x)-\mu_0(x))  \hat d(x)  \\
	E[\hat v | X=x] & = & E[Y | W=\hat d(x), X=x]  \\
	& = & \mu_0(x)(1-\hat d(x)) + \mu_1(x)\hat d(x)
\end{array}
\]

Consider two policies $\hat d_a$ and $\hat d_b$ and their respective estimated values $\hat v_a, \hat v_b$  and gains $\hat g_a, \hat g_b$. The expected difference in value between the two models is 

\[
\begin{array}{rcl}
E[\hat v_a - \hat v_b] 

& = & E[E[\hat v_a | X] - E[\hat v_b|X]] \\

&=& E \left[
	\mu_0(X)(1-\hat d_a(X)) + \mu_1(X)\hat d_a(X) 
      - \mu_0(X)(1-\hat d_b(X)) -  \mu_1(X)\hat d_b(X)
\right] \\

&=& E \left[
	  \hat d_a(X) (\mu_1(X)  - \mu_0(X) ) 
	- \hat d_b(X) (\mu_1(X)  - \mu_0(X) )
\right] \\

&=& E \left[
	  E[\hat g_a | X] ) 
	- E[\hat g_b | X] )
\right] \\

&=& E \left[ \hat g_a - \hat g_b \right] \\

\end{array}
\]

If the estimated value of $\hat d_a$ is larger than than of $\hat d_b$, we expect the relationship between their gains to be the same. Since this relation holds between any two policies $\hat d_a$ and $\hat d_b$, the policy that maximizes gain also maximizes value.

\end{proof}

To our knowledge, this is the first result that demonstrates the link between gain and decision-theoretic value and demonstrates why maximizing gain works in practice. 

It is also interesting that the expected gain of a decision rule is equal to the inner product of that rule with the true treatment effect. Gain is a special case of equation \ref{te-error} where $l(\hat \tau (x_i), \check \tau_i) = -\check \tau_i  I(\hat \tau (x_i) > 0)$ and $\check\tau_i = y_i^{\dagger}$. This reveals a connection between maximizing value and minimizing an error in estimated treatment effect(!). It may be possible to use different estimators of $\check\tau_i$ with this loss, e.g. the matching estimator from section \ref{match-mse}.

\subsection{Consistency and unbiasedness of $\check e$}

\subsubsection{Consistency under squared-error loss}

Here we show for the first time that any unbiased and consistent estimate $\check\tau$ of the treatment effect calculated on the validation set can be used for estimation of the generalization error under squared-error loss. 

\begin{lemma}
If $\check\tau$ is unbiased for $\tau$, then 
\[
E\left[\frac{1}{|\mathcal{V}|}\sum_{i \in \mathcal{V}}^{|\mathcal{V}|}  (\hat \tau(X_i) - \check \tau_i)^2\right] = E[(\hat\tau(X) - \tau(X))^2] + E[(\check\tau(X) - \tau(X))^2]
\]
\end{lemma}

\begin{proof}
\begin{equation}
	E\left[ \frac{1}{|\mathcal{V}|}\sum_{i \in \mathcal{V}}^{|\mathcal{V}|}  (\hat \tau(X_i) - \check \tau_i)^2 \right]  
	= 
	E[ (\hat \tau(X) - \tau + \tau  - \check \tau)^2 ] \\
\label{expected-plug-in-mse}
\end{equation}

The quantity in the sum can be expressed as
\[
E[ (\hat \tau(X) - \tau)^2] + 2E[(\hat \tau(X) - \tau)(\tau - \check\tau)] + E[(\tau - \check\tau)^2]
\]

Using the law of total expectation, we rewrite the second term as $E[E[(\hat \tau(x_i) - \tau(x_i)(\tau(x_i) - \check\tau_i)|X=x_i]]$. Since $\check\tau_i$ and $\hat\tau(x_i)$ are independent, this factors to $E[E[\hat \tau(x_i) - \tau(x_i)|X=x_i]E[\tau(x_i) - \check\tau_i|X=x_i]]$, which is $0$ because $E[\tau(x_i) - \check\tau_i|X=x_i] = 0$ by unbiasedness. Thus equation \ref{expected-plug-in-mse} becomes

\[
	E\left[ \frac{1}{|\mathcal{V}|}\sum_{i \in \mathcal{V}}^{|\mathcal{V}|}  (\hat \tau(x_i) - \check \tau_i)^2 \right]  
	=
	E[ (\hat \tau(X) - \tau(X))^2] + E[(\tau(X) - \check\tau(X))^2]
\]

\end{proof}

The expected estimated error when we use $\check\tau$ as an estimate for $\tau$ is the expected error of our model $\hat\tau$ plus the expected error of our plug-in estimate $\check\tau$. Consequently, the estimated error we obtain is likely to be greater than the true generalization error of our model. However, our estimate can still be used to select between treatment effect models since the surplus error $E[(\tau(X) - \check\tau(X))^2]$ does not depend on the model $\hat \tau(X)$. 

\begin{theorem}
If $\check\tau_i$ is unbiased and consistent for $\tau_i$, then $\underset{m}{\emph{argmin}}\frac{1}{|\mathcal{V}|}\sum_{i \in \mathcal{V}}^{|\mathcal{V}|}  (\hat \tau_m(x_i) - \check \tau_i)^2$ is a consistent estimator of $\underset{m}{\emph{argmin}} \ E[(\hat\tau_m(X) - \tau(X))^2]$.
\end{theorem}

\begin{proof}

($^*$waves hands$^*$) The value of the estimated error for a model $\hat\tau_m$ is $\check e_m = \frac{1}{|\mathcal{V}|}\sum_{i \in \mathcal{V}}^{|\mathcal{V}|}  (\hat \tau_m(x_i) - \check \tau_i)^2 $. As $n$ goes to infinity, this should converge in probability to its expected value, which is a constant $e_m + d$, where $e_m$ is the true generalization error $E[(\tau(X) - \hat\tau(X))^2]$ and $d=E[(\tau(X) - \check\tau(X))^2]$ is a constant that doesn't depend on the model. We compare two models $\hat\tau_A$ and $\hat\tau_B$ on the basis of their estimated error $\check e_A$ and $\check e_B$. In the limit, $\check e_A > \check e_B \iff  e_A + d >  e_B + d \iff  e_A >  e_B$ ($^*$waves hands$^*$) 

\end{proof}

\subsubsection{Unbiasedness under multiplicative loss}

\begin{theorem}
For losses of the form $l(\check\tau(x), \hat\tau(x)) = \check\tau(x) f(\hat\tau(x))$, if $E[\check\tau(X)|X] = \tau(x)$, then $E[\check e] = E [  l(\tau, \hat\tau) ]$.
\end{theorem}

\begin{proof}

\[
\begin{array}{rcl}
	E[\check e] & = & E \left[ \dfrac{1}{| \mathcal V |} \sum_{i \in \mathcal V} \check\tau(X) f(\hat\tau(X)) \right] \\
	& = & E \left[  E[ \check\tau(X) f(\hat\tau(X)) | X] \right] \\
	& = & E \left[  E[ \check\tau(X)|X] f(\hat\tau(X))] \right] \\
	& = & E \left[  \tau(X) f(\hat\tau(X))] \right] \\
	& = & E [  l(\tau, \hat\tau) ]
\end{array}
\]

\end{proof}

We can therefore generalize gain, using any unbiased estimator of $\tau$ instead of $y^{\dagger}$. By an argument similar to that of theorem \ref{gain-value}, maximizing any form of generalized gain will maximize value in expectation.

\subsubsection{Implications}

If squared-error loss is used, only consistency in estimating a difference in generalization errors can be guaranteed, even with an unbiased estimator $\check\tau$. In other words, model selection with these approaches is not guaranteed to work (on average) unless the number of individuals in the validation set is large.

Our results do guarantee that the estimate of generalization error will be unbiased if $\check\tau$ is unbiased for $\tau$ and a multiplicative loss is used. Unbiasedly maximizing the decision-theoretic value of a treatment policy is justifiable in its own right. It is convenient that minimizing generalization error is equivalent to maximizing decision-theoretic value using $l(\check\tau(x), \hat\tau(x))  = \check\tau I(\hat\tau > 0)$ and any unbiased $\check\tau$.


% \section{Variants}
\label{variants}

\emph{Mike's IV estimator and friends go here}

\subsection{Matching on estimated treatment effect}

\label{te-match}

Another idea is to match or stratify individuals in the test set by their estimated treatment effect. After doing so, we assess how well the average estimate compares to the difference in test set outcomes within the strata or matched pair \emph{[personal correspondence w/ Rob]}. Regardless of the details, the idea is to construct an estimator $\check\tau(x_i)$ such that 

\[
E[\check\tau(x_i)] = E[(Y_i - Y_j) | W_i = 1, W_j=0, \hat\tau(X_j) = \hat\tau(x_i)]
\]

Here one can imagine that individuals $i$ and $j$ are a pair matched on the estimated treatment effect such that $\hat\tau(x_i) = \hat\tau(x_j)$. In the case of a randomized experiment, the right-hand-side of the above becomes $E[ \tau(X_j) | \hat\tau(X_j) = \hat\tau(x_i)]$. Unless $\hat\tau(x) = \tau(x)$, this quantity is not generally consistent or unbiased for $\tau(x_i)$ so we have no reason to believe that using it as $\check \tau(x_i)$ should result in a useful estimator of generalization error. Our simulations confirm this result.

% why we can't use a model for \check \tau (must be unbiased)
% \section{AUC-style selection methods}

\label{other}

There are a few treatment effect model selection approaches that we have not examined: the direct marketing literature documents the use of various functions of the ``uplift curve'' \cite{Gutierrez:2016tq}, while \citet{vanKlaveren:2018gg} propose what they call the concordance-for-benefit statistic. 

Recall that the decision policy can be parametrized with a cutoff $k$: $\hat d(x,k) = I(\hat\tau>k)$. We can therefore also calculate the gain at $k$, $\hat g(k)$, which is the gain obtained when individuals with an estimated treatment effect greater than $k$ are treated. Researchers in direct marketing often rely on \emph{uplift curves} (also called gain curves or cumulative gain charts) \ to aid in model selection. The uplift curve plots gain at $k$ against the percentage of individuals in the validation set recommended for treatment at $k$. These curves are evaluated either heuristically, using the maximum value over $k$, or, most commonly, using the area under the curve \cite{Gutierrez:2016tq}. The area under the gain curve is called the \emph{Qini coefficient}.

\citet{vanKlaveren:2018gg} propose the concordance-for-benefit statistic to select among treatment effect models. Briefly, each individual $i$ in the validation set is matched without replacement according to estimated treatment effect with another individual in the validation set $\bar{i}$ ($\hat\tau(x_i) \approx \hat\tau(x_{\bar i})$) with the opposite treatment $w_i \ne w_{\bar i}$. The total number of pairs is $J$. For each pair of matched patients $j$, the difference in outcomes between the treated and untreated individuals is recorded: $\delta_j = (2w_i - 1)(y_i -y_{\bar i})$, along with the predicted treatment effect for both patients $t_j = \hat\tau(x_i)$. The concordance-for-benefit (or c-for-benefit) statistic is

\[
\hat c = \frac{\sum_j^J \sum_{j' > j}^J (1 - \text{sign}((\delta_{j'} - \delta_j)(t_{j'} - t_j))) }{(J^2-J)}
\]

This is the proportion of pairs-of-pairs in which the difference in predicted treatment effects is discordant with the difference in ``observed'' treatment effects in the validation set. This assesses the overall ranking of individuals with respect to their estimated treatment effects.

Both of these approaches are inspired by different interpretations of the area under the receiver operating characteristic curve statistic (AUC), which is both the area under a parametric curve indexed by a classification cutoff (i.e. $k$) and an estimate of the proportion of pairs expected to have discordant classifications \cite{Hanley:1982cz}. We conjecture that there is a theoretical relationship between these approaches and that they are related to decision value, but do not provide results to that effect. However, we do test both of these methods in simulation, along with value-at-k AUC, which is equivalent to the Qini coefficient, but using $\hat{v}(k)$ instead of $\hat{g}(k)$.

%%%%%%%%%%%%%%%%%%%%%%%%%%%%%%%%%%%%%%%%%%%%%%%%%%%%%%%%%%%%%%%%%%%%%%%%%%%%%%%%%

\begin{comment}

Recall that the decision policy can be parametrized with a cutoff $k$: $\hat d(x,k) = I(\hat\tau>k)$. We can therefore also calculate the gain at $k$, $\hat g(k)$, which is the gain obtained when individuals with an estimated treatment effect greater than $k$ are treated. Researchers in direct marketing often rely on \emph{uplift curves} (also called gain curves or cumulative gain charts) \ to aid in model selection. The uplift curve plots gain at $k$:

\[
	\hat g(k)  =  \sum_{\mathcal{V}} \dfrac{y_i  \hat d(x_i,k) (2w_i-1)}{p_{w_i}(x_i)} 
\]




Researchers either evaluate these curves heuristically, take the maximum value over $k$, or calculate the \emph{Qini coefficient}, which is the area under the uplift curve \cite{Gutierrez:2016tq}. Given the relationship between gain and value, maximizing these metrics is equivalent to maximizing the equivalent value-based metric. 

\subsubsection{The concordance-for-benefit statistic}

\citet{vanKlaveren:2018gg} propose the concordance-for-benefit statistic to select among treatment effect models. Briefly, each individual $i$ in the test set is matched without replacement according to estimated treatment effect with another individual in the test set $\bar{i}$ ($\hat\tau(x_i) \approx \hat\tau(x_{\bar i})$) with the opposite treatment $w_i \ne w_{\bar i}$. The total number of pairs is $J$. For each pair of matched patients $j$, the difference in outcomes between the treated and untreated individuals is recorded: $\delta_j = (2w_i - 1)(y_i -y_{\bar i})$, along with the predicted treatment effect for both patients $t_j = \hat\tau(x_i)$. The concordance-for-benefit (or c-for-benefit) statistic is

\[
\hat c = \frac{\sum_j^J \sum_{j' > j}^J (1 - \text{sign}((\delta_{j'} - \delta_j)(t_{j'} - t_j))) }{(J^2-J)}
\]

This is the proportion of pairs-of-pairs in which the difference in predicted treatment effects is discordant with the difference in ``observed'' treatment effects in the test set. 

At first glance, this resembles the ineffective treatment effect matching approach we describe in section \ref{te-match}. The difference is that the c-for-benefit does not assess the disparity between ``observed'' and estimated treatment effects- it assesses the quality of the ranking according to the estimated treatment effect. The c-for-benefit is an estimate of:

\[
P(Y_i - Y_{\bar i} > Y_{i'} - Y_{\bar i'} | W_i = W_{i'} = 1, W_{\bar i} = W_{\bar i'} = 1, \hat\tau(X_i) = \hat\tau(X_{\bar i}), \hat\tau(X_{i'}) = \hat\tau(X_{\bar i'}), \hat\tau(X_i) > \hat\tau(X_{i'}))
\]

The c-for-benefit may be related to value and gain. To see this, we simplify the procedure: instead of matching pairs, we bin individuals into $J$ bins, each with approximately equal treatment effect: $S_j = \{i | l_j \le \hat\tau(x_i) < u_j\}$ where $u_j = l_{j+1}$, $l_1 = \underset{i \in \mathcal{V}}{\text{min}} \ \hat\tau(x_i)$, and $u_J = \underset{i \in \mathcal{V}}{\text{max}} \ \hat\tau(x_i)$. To capture the spirit of the original definition, we let $t_j$ be the average estimated treatment effect in each bin and we calculate 

\[
\delta_j = \sum_{i \in S_j} \frac{y_i w_i}{p_i} - \sum_{i \in S_j} \frac{y_i (1-w_i)}{(1-p_i)}
\]

so that $\delta_j$ is an unbiased estimate of the treatment effect among individuals in bin $S_j$. Note that this definition of $\delta_j$ is identical to our definition of the generalized gain $\hat g$ if the policy is $\hat d(x_i) = I(i \in S_j)$. In the case where there are only two bins $S_1 = \{i | \hat\tau(x_i) < 0\}$ and $S_2 = \{i | 0 \le \hat\tau(x_i)\}$, the binned c-for-benefit statistic is

\[
\hat c^{^{split}} = \frac{1}{2} \left( 1 - \text{sign}((\delta_2 - \delta_1)(t_2 - t_1)) \right)
\]

By construction, $t_2 - t_1 > 0$. We also have that $\delta_1 = \hat g(1-\hat d(x))$ and $\delta_2 = \hat g(\hat d(x))$, where $\hat d(x) = I(\hat\tau(x) > 0)$. Therefore

\[
\hat c^{split} = \frac{1}{2} (1 - \text{sign}(\hat g(\hat d) - \hat g(1- \hat d)))
\]

\begin{theorem} 
As the size of the test set goes to infinity, the model that maximizes gain (and thus value) is one of the models that minimizes $\hat c^{split}$.
\end{theorem}

\begin{proof}
Recall that $E[\hat g(\hat d)] = E[\tau(X) \hat d(X)]$. Now note that
\[
\begin{array}{rcl}
	E[\hat g(1-\hat d)] &=& E[E[\hat g(1-\hat d)|X]] \\
	&=&  E[\tau(X)(1-\hat d(X))] \\
	&=&  E[\tau(X) - \tau(X) \hat d(X)] \\
	&=&  \bar \tau - E[\tau(X) \hat d(X)] \\
	&=&  \bar \tau - E[\hat g(\hat d)]
\end{array}
\]	
	
Take two decision rules $\hat d_a$ and $\hat d_b$. 
\[
\begin{array}{rcl}
E[\hat c^{split}_a - \hat c^{split}_b] & = & - E[\text{sign}(\hat g_a - \hat g_{1-a})] + E[\text{sign}(\hat g_b - \hat g_{1-b})] \\
\end{array}
\] 

\emph{*waves hands*} In the limit, $\hat g_m - \hat g_{1-m}$ should converge to $2E[\hat g_m] - \bar \tau$, so we have the following cases:

\[
E[\hat c^{split}_a - \hat c^{split}_b] \rightarrow \left\{
\begin{array}{rcl}
2 & \text{if} & E[\hat g_b] > \bar \tau \text{ and } E[\hat g_a] < \bar \tau \\
0 & \text{if} & E[\hat g_b] > \bar \tau \text{ and } E[\hat g_a] > \bar \tau \\
0 & \text{if} & E[\hat g_b] < \bar \tau \text{ and } E[\hat g_a] < \bar \tau \\
-2 & \text{if} & E[\hat g_b] < \bar \tau \text{ and } E[\hat g_a] > \bar \tau \\
\end{array} \right.
\] 

Consequently, $E[\hat c^{split}_a] > E[\hat c^{split}_b]$ only if $E[\hat g_a] < E[\hat g_b]$ and $E[\hat c^{split}_a] < E[\hat c^{split}_b]$ only if $E[\hat g_a] > E[\hat g_b]$.

\end{proof}

To illustrate this relationship we have had to simplify the c-for-benefit, which may fundamentally change its nature. The original pair-matching procedure is adaptive, whereas our binned procedure uses an a-priori split.

\end{comment}
\section{Simulations}
\label{simulations}

\subsection{Overview}

We demonstrate the utility of these approaches using simulations. Each simulation is defined by a data-generating process with a known effect function, which allows us to compute true test set errors. Each run of each simulation generates a dataset, which we split into training and test samples. We perform $5$-fold cross validation on the training data and compute each model selection metric on each validation fold. I.e. for each training fold, we estimate $M$ different treatment effect functions $\hat\tau_m$ and for each of those we calculate each metric using the data in the corresponding validation fold. The metrics are averaged for each model across all folds. Each model selection approach then selects the model among the $M$ models that minimizes (or maximizes, when appropriate) its corresponding metric. The models selected by each approach on the basis of cross-validation are refit on the full training data and applied to the test set. The test-set treatment effect estimates of each model are compared to the known effects to calculate the true cost of using each approach for model selection. Each simulation is repeated multiple times. All of the code used to set up, run, and analyze the simulations is feely available on \href{https://github.com/som-shahlab/ITE-model-selection}{github}.

\subsubsection{Data-generating processes and sampling}

We use the sixteen simulations from \citet{Powers:2017wd}, each of which we repeat $50$ times. In each repetition, $1000$ samples are used for training and validation and $2000$ are used for testing. 

\subsubsection{Models}

We fit several treatment effect models to each simulated dataset. We limit ourselves to two-model conditional mean regressions: given a model specification, we fit two separate models to predict the outcomes of the individuals who were treated and not treated and take the difference in predicted outcomes as the estimated individual treatment effect. This is so that we can calculate outcome prediction MSE as a selection metric for comparison. As a result we do not use any models that directly estimate the individual treatment effect (e.g. causal forests). The models we use are gradient boosted trees (number of trees ranging from $1$ to $500$, tree depth of $3$, shrinkage of $0.2$ and minimum $3$ individuals per node) and elastic nets ($\alpha=0.5$, $\lambda \in [e^{-5}, e^2]$). These models give us a range of high-performing linear and nonlinear models to select among. For efficiency, we only consider combinations of $\hat\mu_1$ and $\hat\mu_0$ that were fit using the same method with the same hyperparameters. This constraint need not be enforced in practice (i.e. $\hat\mu_0$ could be fit using a linear model and $\hat\mu_1$ fit using a random forest). All models are fit using the caret R package.

\subsubsection{Model selection approaches}

The following metrics are used to select among models in each simulation:

% This table is terrible. Need to 1) get Type to be multicolumn 2) fix vertical spacing b/t lines 3) make Ref section much thinner

\begin{center}
\begin{tabu}{|m{6cm}|c|c|}
	\hline
	 \rowfont[c]{\bfseries} Metric & Reference section & Type \\
	 \hline
	 Random & NA &  NA \\
	 \hline
	 Covariate matched-pairs ITE MSE & \ref{match-mse} & ITE loss \\
	 Transformed outcome ITE MSE & \ref{match-mse} & ITE loss \\
	 Covariate matched-pairs ITE decision cost &  \ref{sec:gain-value} & ITE loss (Value) \\
	 Transformed outcome ITE decision cost (neg. generalized gain) &  \ref{sec:gain}, \ref{sec:gain-value} & ITE loss (Value) \\
	 \hline
	 Traditional gain &  \ref{sec:gain} & Value \\ 
	 Decision value &  \ref{sec:value} & Value \\
	 \hline
	 C-for-benefit &  \ref{other} & AUC \\
	 Qini coefficient (gain-at-k AUC) &  \ref{other} & AUC \\
	 Value-at-k AUC &  \ref{other} & AUC \\
	 \hline
	 Est. treatment effect strata ITE MSE &  \ref{te-match} & Other \\
	 \hline
	 Outcome prediction MSE &  \ref{sec:pred-error} & Outcome loss \\
	 \hline
\end{tabu}
\end{center}

Taking the model that minimizes (or maximizes, when appropriate) one of these metrics on average across validation folds defines a model selection approach. 

We split the approaches into several categories based on the theoretical motivation for each. The ITE loss approaches all minimize metrics of the form of equation \ref{te-error}. The ``value'' approaches are those that have some theoretical equivalence to maximizing the decision value of a model. The AUC-style approaches are briefly described in section \ref{other}. 

Note that the ``random'' metric assigns a random number to each model, which means that the model selected by that metric is chosen uniformly at random from the available models. 

In simulations with biased treatment assignment, we estimate a propensity score with logistic regression. That estimated propensity is used in place of the true value in calculating all metrics that require a propensity score.

\subsubsection{Evaluation metrics}

Let the model $\hat\tau_m$ selected by optimizing metric $h$ in cross-validation be written as $\hat\tau^{*_h}$. 

We are interested in the quantities

\[
\tau MSE_h = E[ (\hat\tau^{*_h} (X) - \tau(X))^2 ]
\]

and 

\[
v_h = E[ Y| W =\hat d^{*_h} (X)]
\]

which we unbiasedly estimate in a large test set $\mathcal{T}$ via

\begin{equation}
\label{true-mse}
\tau MSE^{(\mathcal{T})}_h = \frac{1}{|\mathcal{T}|}\sum_{i \in \mathcal{T}} (\hat\tau^{*_h} (x_i) - \tau(x_i))^2
\end{equation}

and 

\begin{equation}
\label{true-value}
v^{(\mathcal{T})}_h = \frac{1}{|\mathcal{T}|}\sum_{i \in \mathcal{T}} \mu_{\hat d^{*_h}(x_i)}(x_i)
\end{equation}

Where $\hat d^{*_h}(x_i) = I(\hat\tau^{*_h}(x_i) > 0)$ as before. 

$\tau MSE_{h}^{(\mathcal{T})}$ calculates how well the selected model estimates the treatment effect for individuals in the test set. $v^{(\mathcal{T})}$ is the decision value of applying the treatment policy $\hat d(x)$ derived from each selected model $\hat \tau (x)$ to the individuals in the test set.

These are both useful metrics, although only the first ($\tau MSE$, sometimes called ``precision in estimating heteorgenous effects'', or PEHE) has typically been used in simulation studies. To see why the test set decision value is also important, consider two models ($A$ and $B$) that estimate the same treatment effect for all individuals, except two ($i=1$ and $i=2$). Let $\tau(x_1) = \tau(x_2) = 0.1$, i.e. both individuals would benefit from the treatment in reality. Model $A$ estimates $\hat\tau_A(x_1) = -0.1$ and $\hat\tau_A(x_2) = 0.1$. In other words, it incorrectly suggests not treating individual $1$, although the absolute difference $|\hat\tau_A(x_1)-\tau(x_1)| = 0.2$ is quite small, so it is not heavily penalized by $\tau MSE$. Model $B$ estimates $\hat\tau_B(x_1) = 0.1$ and $\hat\tau_B(x_2) = 100$. Model $B$ correctly suggests the treatment for both individuals, but the absolute difference $|\hat\tau_B(x_2)-\tau(x_2)| = 99.9$ is large and is heavily penalized by $\tau MSE$. Often, what we want is a model that correctly assigns treatment to the individuals who stand to benefit from it. Using $\tau MSE$ in this case would favor model $A$ even though it leads to the mistreatment of more individuals than model $B$ does. However, $\tau MSE$ is still a useful metric. There may be cases where a researcher is interested in the precise magnitude of the effect for each individual, perhaps so that scarce resources can be allocated most effectively. 

In our simulations, we also calculate the optima of both $v^{(\mathcal{T})}$ and $MSE^{(\mathcal{T})}$ over all the computed models $\hat\tau_m \in \{\hat\tau_1 \dots \hat\tau_M\}$:

\[
\tau MSE^{(\mathcal{T})}_{*} = \underset{m}{\text{min}} \ \ \frac{1}{|\mathcal{T}|}\sum_{i \in \mathcal{T}} (\hat\tau_m (x_i) - \tau(x_i))^2
\]

\[
v^{(\mathcal{T})}_{*} = \underset{m}{\text{max}} \ \ \frac{1}{|\mathcal{T}|}\sum_{i \in \mathcal{T}} \mu_{\hat d_m(x_i)}(x_i)
\]

These quantities represent the best possible performance that any model selection method could achieve given the available models in a given simulation. We use these quantities as baselines.

%%%%% ALTERNATIVELY: use % of value captured. For any DGP, the max test-set value is calculated using d=I(\tau < 0). Call that v_min. v_max is calculated using I(\tau > 0). v_model uses the I(\hat\tau > 0). We always have v_min <= v_model <= v_max. Let % value be (v_model - v_min)/(v_max - v_min)

\begin{comment}
We also calculate these using the true model $\hat\tau = \tau$, which represents the best achievable performance without specifying any models a-priori:

\[
\tau MSE^{(\mathcal{T})}_{**} = 0
\]

\[
v^{(\mathcal{T})}_{**} = \frac{1}{|\mathcal{T}|}\sum_{i \in \mathcal{T}} \mu_{ d(x_i)}(x_i)
\]

We use these minima to calculate the improvements obtained by each model selection approach relative to optimal baselines, which allows for comparison between different simulations. The relative values of $\tau MSE^{(\mathcal{T})}_{h}$ and $C^{(\mathcal{T})}_{h}$ within one test set and one set of models are

\[
	R\text{-} \tau MSE^{(\mathcal{T})}_{h} = 
	\frac{\tau MSE^{(\mathcal{T})}_{h} - \tau MSE^{(\mathcal{T})}_{**}}{\tau MSE^{(\mathcal{T})}_{*} - \tau MSE^{(\mathcal{T})}_{**}}
\]

\[
	R\text{-}  v^{(\mathcal{T})}_{h} = 
	\frac{ v^{(\mathcal{T})}_{h} -  v^{(\mathcal{T})}_{**}}{ v^{(\mathcal{T})}_{*} -  v^{(\mathcal{T})}_{**}}
\]

The closer to zero these are, the better the performance of the model selection method.
\end{comment}

\subsection{Results}

\begin{figure}
\centering
\includegraphics[width=\textwidth]{value-sim-comparison}
\caption{Test-set decision value ($v_h$) of the model selected by each approach (y-axis) in each simulation (x-axis) relative to the test-set decision value of the true best model ($v_*$), averaged over each repetition of the simulation. Selection approaches are ranked top-to-bottom by averages of this quantity taken across all simulations. Values closer to zero indicate that the model selection approach choses better models on average.}
\end{figure}

\begin{figure}
\centering
\includegraphics[width=\textwidth]{tmse-sim-comparison}
\caption{Test-set treatment effect mean-squared-error ($\tau MSE_h$) of the model selected by each approach (y-axis) in each simulation (x-axis) relative to the test-set  treatment effect mean-squared-error of the true best model ($\tau MSE_*$), averaged over each repetition of the simulation. Selection approaches are ranked top-to-bottom by averages of this quantity taken across all simulations. Values closer to zero indicate that the model selection approach choses better models on average.}
\end{figure}

ITE loss-minimizing approaches perform well, but those that employ covariate matching are notably worse off in several simulations, likely due to the difficulty of finding good covariate matches in high dimensions. All of the approaches related to value estimation perform similarly, as expected, since the metrics they are optimizing have equivalent differences in expectation (theorem \ref{gain-value}). In some simulations with biased assignment, the traditional formulation of gain does poorly where our generalized version succeeds, presumably because our formulation correctly incorporates the propensity score. Of the AUC-style approaches, the Qini coefficient performs poorly, C-for-benefit performs reasonably, and our proposed value-at-k AUC does well. Stratifying on the predicted treatment effect as described in section \ref{te-match} fares poorly, as predicted. A straightforward application of outcome prediction MSE does very well. Random model selection leads to reasonable performance in many simulations because all of the available models perform only somewhat worse than the best among them. This is reflective of real-world use where poor models or hyperparameter combinations are removed a-priori. 

Relative performance between methods is similar when gauged by test-set decision value and by test-set $\tau MSE$. 

\section{Conclusion}
\label{conclusion}

Although validation-set prediction error is used across the board to select between predictive models, there is no consensus on how to perform model selection for individual treatment effects models. However, general-purpose approaches have been proposed. We describe each of these approaches and demonstrate how many of them work by minimizing a loss between the treatment effect function estimated on the training set and applied on the validation set and a treatment effect directly estimated on the validation set. The validation set estimate serves as a plug-in for the true individual treatment effect of each individual in the validation set. We establish some conditions on the validation set estimator that ensure statistically sound model selection. We also explain methods that are based on optimizing decision value and demonstrate a connection between them and the treatment effect loss-based methods. 

Outcome prediction error, transformed outcome ITE MSE, and 

\bibliographystyle{plainnat}
\bibliography{references}

\end{document}
